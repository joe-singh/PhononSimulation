\documentclass[11pt]{article}
\usepackage{amsmath}
\usepackage{amsthm}
\usepackage{amssymb}
\usepackage{enumitem}
\usepackage[margin=0.5in]{geometry}
\usepackage{siunitx}
\usepackage{graphicx}
\graphicspath{ {./} }
\newcommand{\pic}[0]{\textbf{ATTACH PICTURES}}
\newcommand{\DO}{\textbf{DO!!}}
\newcommand{\sig}{On signature sheet.}
\newcommand{\e}[1]{\times 10^{#1}}
\newcommand{\vol}{\si{\volt}}
\newcommand{\mv}{\si{m\volt}}
\newcommand{\amp}{\si{\ampere}}
\newcommand{\hz}{\si{\hertz}}
\usepackage[colorlinks]{hyperref}
\usepackage[utf8]{inputenc}

\begin{document}
\title{Phonon Surface Physics and Monte Carlo Techniques in the SuperCDMS Experiment.}
\author{Jyotirmai Singh}
\date{}
\maketitle

\section*{Basic Elements}
Phonons are the quantised vibration modes of crystal lattices. They are described by the Hamiltonian \cite{1}

\begin{equation}
H = \sum_i \frac{p_i^2}{2m} + \sum_{i,j} \frac{m\omega^2}{2}(x_i - x_j)^2
\end{equation}

The aim of this work is to add to the CDMS Monte Carlo by integrating new diffusive propagation and anharmonic
down conversion effects that occur at the boundaries of the crystal lattices in which the phonons propagate. 

For the purposes of this Monte Carlo, we follow the earlier precedent from \cite{1} and treat the phonons as non
interacting particles with specific decay and scattering properties in both the bulk and at the surfaces of the crystal. We
will also be primarily concerned with a low temperature range where acoustic phonons can be modeled by the linear 
dispersion relaion $\omega = vk$. The initial frequencies of the phonons are uniformly distributed from $80 \si{G\hertz}$ to $524 \si{G\hertz}$.
This matches the limits set based on the sensitivity of the SuperCDMS experiment. Phonons lower than $80\si{G\hertz}$ will
be not be detected by the TES sensing elements \cite{2} while the high scale is set by considerations relating to surface decay effects that will be
mentioned later. This means the upper bound, and the initial distribution, for the three modes is as follows
\begin{table}[!h]
\begin{center}
\begin{tabular}{|c|c|c|} \hline
Mode & Upper Frequency [$\si{T\hertz}$] & Initial Proportion \\  \hline
Longitudinal & $6.2$ & $10\%$ \\ \hline
Fast Transverse & $3.7$ & $35\%$ \\ \hline
Slow Transverse & $2.4$ & $55\%$\\ \hline
\end{tabular}
\end{center}
\caption{The upper frequencies for the initial phonon distribution, and the initial phonon distribution. The initial distribution of phonon modes is as given in both \cite{1} and \cite{2}.}
\end{table}

Computationally, the Monte Carlo model follows a Discrete Event Simulation (DES). We opt for this because simulating discrete time steps of
a given fixed small length is computationally inefficient as a large amount of simulated steps will not be of interest. Instead, we simulate only 
events of interest. To determine which event occurs next, we first calculate a characteristic time step for each process given by 
$\tau = -\frac{ln(r)}{R}$ where $r$ is randomly distributed from $0$ to $1$ and $R$ is the rate of the process in question. In addition, we 
calculate a time associated with hitting the boundary. This time is calculated assuming the phonon travels at its current velocity and thus a simple
calculation is used to find the distance to the closest point on the boundary in the direction of the velocity vector, and then $\tau = \frac{l}{v}$
where $l$ is the calculated distance and $v$ is the current phonon speed. Whichever time is the smallest from these is the process that 
is chosen to occur next, and the  phonon's time value is incremented by that smallest time. The rest of this note will address exactly what processes
we simulate in bulk and on the boundary and how these rates are calculated.

\section*{Bulk Interactions}

The bulk interactions are dominated by two main processes. These are isotopic scattering and bulk anharmonic decay. 

\subsection*{Isotopic Scattering}
As explained in \cite{1}, phonons scatter off mass defects in the crystal. The rate for this process is given by
\begin{equation}
\Gamma_I = B\nu^4\label{eq:2}
\end{equation}

Where $B = 2.43 \times 10^{-42}$ in Silicon. For the purposes of the simulation, this process involves the incoming phonon spontaneously 
changing its propagation direction in a uniform manner. The new uniform velocity is picked in spherical coordinates, with the azimuthal angle a
uniform choice between $0$ and $2\pi$ and the polar angle chosen from a $\sin(\theta)$ distribution for $\theta \in [0, \pi]$.

\subsection*{Anharmonic Decay}
A second, weaker process in the bulk decay spectrum is anharmonic decay, through which a phonon will down convert to two lower energy
phonons. This occurs only for longitudinal phonons, and the possible decay modes are either $L\rightarrow T+T$ or $L\rightarrow T+L$.
The rate of this interaction is given by 

\begin{equation}
\Gamma_{A} = A\nu^5\label{eq:3}
\end{equation}

Where $A = 7.41 \times 10^{-56} \ \si{\second^4}$ for Silicon \cite{1}.

\subsubsection*{$L\rightarrow L'+T$:}
The decay rate for this process is given by \cite{3}

\begin{equation}
\Gamma_{L\rightarrow L'T} \sim \frac{1}{x^2}(1-x^2)^2\left[(1+x)^2 - \delta^2 (1-x)^2\right]\left[1 + x^2 - \delta^2(1-x)^2\right]^2
\end{equation}

Where $x = \frac{\omega_{L'}}{\omega_L}$ is the ratio of the final longitudinal frequency to the initial, $\delta = \frac{v_l}{v_t}$ is the ratio of
the longitudinal phonon velocity in the crystal to transverse phonon velocity and we must have that $\frac{\delta - 1}{\delta + 1} < x < 1$. This 
distribution is used to sample the frequency of the produced longitudinal phonon, and then the transverse phonon frequency is simply set to be
the difference of the initial frequency and the final longitudinal frequency to respect conservation of energy. 

4-momentum conservation also gives us the angles relative to the initial velocity vector of the final phonons:

\begin{equation}
\cos(\theta_{L'}) = \frac{1 + x^2 - \delta^2(1-x)^2}{2x} 
\end{equation}
\begin{equation}
\cos(\theta_T) = \frac{1 - x^2 + \delta^2(1-x)^2}{2\delta(1-x)}
\end{equation}

We plot the (unnormalised) probability distributions for this process below. 

\begin{figure}[!h]
\begin{center}
\includegraphics[scale=.5]{./bulk_anharmonic_pdfs.png}
\end{center}
\caption{The probability densities (unnormalised) of the L (blue) and T(green) phonons of the LLT mode and of the T phonon in the LTT mode (red).}
\end{figure}

\subsubsection*{$L\rightarrow T + T$:}
The decay rate for this process is also given in \cite{3}:

\begin{equation}
\Gamma_{L\rightarrow TT} \sim (A + B\delta x - Bx^2)^2 + \left[Cx(\delta - x) - \frac{D}{\delta - x}\left(x - \delta - \frac{1-\delta^2}{4x}\right)\right]
\end{equation}

Where the constant $A, B, C, D$ are related to the properties of the crystal and can be found in both \cite{2} and \cite{3}. Here, the limits
for the pdf support are $\frac{\delta - 1}{2} \leq x \leq \frac{\delta + 1}{2}$, and the variable itself is now defined as $x = \delta\frac{\omega}
{\omega_0}$. The corresponding angular equations can be obtained once again from
4-momentum conservation 

\begin{equation}
\cos(\theta_1) = \frac{1 - \delta^2 + 2x\delta}{2x}
\end{equation}
\begin{equation}
\cos(\theta_2) = \frac{1 - x\cos(\theta_1)}{d - x}
\end{equation}

These equations are slightly different to those from \cite{1}. The equations presented there are problematic since they are not strictly bounded
by $\pm 1$. The $\cos(\theta)$ distributions calculated here are shown below, adjusted for the scaling of the $x$ variable so that the distributions
are now functions of just the frequency ratio without the scaling factor.

\begin{figure}[!h]
\centering
\includegraphics[scale=.5]{./LTT_cos_theta.png}
\caption{The distribution of the cosine of the scattering angle for the LTT case.}
\end{figure}

As expected, the distributions are very symmetric since both outputs are transverse phonons and thus neither should have a privileged position with 
respect to the other. We have also put the plots from \cite{2} for comparison. The bounds are determined by those for the LTT support and are
the same as those for the red LTT plot in figure 1 above. Note how the previous equations are not bounded properly in this range and 
demonstrates an asymptote at the point where $\frac{\omega}{\omega_0} = \frac{1}{d} = 0.61$ for silicon.

\pagebreak
Finally, we present (normalised) distributions of the scattering angles of phonons from all interactions:

\begin{figure}[!h]
\centering
\includegraphics[scale=.6]{scattering_angle_distributions.png}
\end{figure}

The LLT distributions match those given in \cite{1}, which serves as a check on our work. The LTT distributions are virtually identical, as expected,
and almost appear to match those shown before, which is strange given the differences in the cosine formulas quoted there from ours. 

\section*{Boundary Interactions}
The key new addition to the Monte Carlo are the interactions at the boundary. Recall that we use DES to simulate our Monte Carlo. If a step terminates
on the boundary then the next step will not choose from the rates for bulk processes but for the following boundary processes:

\begin{enumerate}
\item Specular scattering
\item Lambertian scattering
\item Surface anharmonic decay (SAD)
\end{enumerate}

\subsection*{Calculation of Boundary Interaction Rates}

The calculation of boundary interaction rates was a key new focus of this work. First we present a summary of the calculation process and then 
the actual rates for all 3 interactions themselves. First, we obtain the total scattering rate on the boundary. Klitsner and Pohl studied phonons
in Silicon extensively, including measuring diffusive scattering rates as a function of the phonon temperature \cite{4}. We adopt the same 
conversion between phonon frequency and temperature that they do:

\begin{equation}
\nu = 4.25 \frac{k_b}{h} T = (90 \ \si{G\hertz \ \kelvin^{-1}})\label{eq:10}
\end{equation}

Fig 1. in \cite{4} illustrates their measurement of the total scattering rate, from which we can subtract the total bulk rate -- given by
adding \eqref{eq:2} and \eqref{eq:3}. Figure 3 illustrates this total surface rate.

\begin{figure}[!h]
\centering
\includegraphics[scale=.6]{Total_Diffusive.png}
\caption{The total diffusive surface scattering rate, calculated by subtracting the total bulk rate - including bulk isotopic and anharmonic 
interactions - from the total scattering rate. Red is an applied fit to the total surface scattering rate.}
\end{figure}


Having obtained the total diffusive rate, we now turn it into a probability by dividing this rate by the diffusive scattering rate expected
for a rough surface which would demonstrate maximal diffusive scattering, which is provided in \cite{4} as well. 

We can also use the total diffusive rate to calculate the rate for the Surface Anharmonic Decay process. We model the total
rate as a sum of that on the bare surface and that on the covered surface:

\begin{equation}
\Gamma_T = \Gamma_B + \Gamma_C 
\end{equation}

We can model these terms as follows:

\begin{equation}
\Gamma_C = \frac{K_C}{t_R}\frac{A_C}{A_T}
\end{equation}

Where $K_C$ is the probability of an anharmonic decay on the covered surface, $t_R$ is the reverbration time for the phonon and $A_C$ and $A_T$
are covered surface area and total surface area. The reverbration time can be calculated using the Sabine equation, which gives the average scattering
length of the phonons assuming ballistic propagation:

\begin{equation}
t_R = \frac{\langle c \rangle}{l_{av}} = \frac{c A_T}{4V}
\end{equation}

Where $\langle c \rangle$ is the average phonon speed in the material across all modes, which for Si is $5.93 \times 10^{5} \ \si{c\meter
\second^{-1}}$ \cite{4} 
and $V$ is the total sample volume.

Which means that 

\begin{equation}
\Gamma_C = K_C \frac{\langle c \rangle}{\frac{4V}{A_T}}\frac{A_C}{A_T} = K_C \frac{\langle c \rangle A_C}{4V}
\end{equation}
 
 By the same logic we can derive the rate for the uncovered boundary
 
 \begin{equation}
 \Gamma_B = K_B \frac{\langle c \rangle A_B}{4V}
 \end{equation}
 
 Which gives a total surface anharmonic decay rate of 
 
 \begin{equation}
 \Gamma_{SAD} = K_C \frac{\langle c \rangle A_C}{4V} + K_B \frac{\langle c \rangle A_B}{4V}
 \end{equation}
 
 Or, using the fact that $A_T = A_B + A_C$ to eliminate $A_B$
 
 \begin{equation}
 \Gamma_{SAD} = K_B \frac{\langle c \rangle A_T}{4V} + (K_C - K_B)\frac{\langle c \rangle}{4} \left(\frac{A_C}{V}\right)
 \end{equation}
 
 We have put it in this form to be able to form a direct comparison with the work of Trumpp and Eisenmenger, who studied the reverbration 
 times of phonons as a function of the ratio of the covered surface area to total volume $\left(\frac{F_C}{V}\right)$ \cite{4}. Our goal is to determine
 $K_B$, the probability of a decay on the bare surface. Furthermore, in the 
 above expression we recognise $\frac{F_T}{4V}$ as the inverse of the mean free length, which we denote $\Lambda_F$.  Trumpp and Eisenmenger 
 note that the mean free path for a ratio of 0.1 is $20\si{c\meter}$. There is a subtlety here in that our mean free path is the distance until 
 the first impact, while for Trumpp and Eisenmenger the free path represents an `effective' path until the phonon is finally absorbed. 
 
 Unfortunately they do not supply the full details of their samples, such as the exact dimensions of the samples they use which makes an 
 explicit calculation difficult. Nevertheless, we estimate that the mean free length until the first impact must be less than $20\si{c\meter}$ and
 using further geometric arguments we can reason it should be on the order of $1 \si{c\meter}$. Thus, approximating $\frac{4V}{F_T}$ with this
 value, we can estimate the value of $K_B$ using the y intercept of the plot in their paper. Using a value of $\Gamma(0) = \frac{1}{65 \times 10^{-6}}$, we obtain the probability for phonon anharmonic decay on the surface.
 
 \begin{equation}
 K_{B_0} = \Gamma(0) \times \frac{\Lambda_F}{\langle c \rangle} = 0.0259
\end{equation}  
 
Trumpp and Eisenmenger studied phonons of a fixed frequency, so the above $K_B$ is valid for this frequency ($280\si{G\hertz}$). However,
there are strong reasons to believe that this probability should scale as $\omega^5$ in at least a limited low temperature regime. Thus we 
may calculate the surface anharmonic probability as follows:

\begin{equation}
K_{SAD}(\nu) = K_{B_0} \left(\frac{\nu}{280 \times 10^{9}}\right)^5
\end{equation}
 
Having obtained the probability for surface anharmonic decay and the total diffusive scattering rate, we can get the lambertian scattering rate
by simply subtracting the former from the latter. In the same manner, we can get the specular scattering rate by subtracting the total diffusive 
rate from 1. In this manner, we can obtain all relevant rates, presented in the plot below:

\begin{figure}[!h]
\centering
\includegraphics[scale=.6]{All_Rates.png}
\end{figure}
 
For the purposes of this work, we just require a fit that accurately replicates these numbers rather than some fit with sound theoretical motivations.
As such, we choose the smallest degree polynomial fit with a good fit - characterised by a chi squared test - to model the probability. We chose the 
smallest degree to avoid overfitting to the data.
 Doing so 
yields the following final boundary event probabilities:

\begin{table}[!h]
\centering
\begin{tabular}{|c|c|} \hline
Process Type & Probability (frequency in GHz) \\ \hline
Lambertian  & $-2.98\e{-11} f^4 + 1.71\e{-8} f^3 - 2.47\e{-6} f^2 + 7.83\e{-4} f + 5.88\e{-2}$ \\ \hline
SAD & $1.51\e{-14} f^5 + 9.27\e{-18} f^4 - 6.28\e{-15} f^3 + 2.01\e{-12} f^2 - 2.28\e{-10}  f + 6.02\e{-9}$ \\ \hline
Specular & $2.9\e{-13} f^4 + 3.1\e{-9} f^3 - 3.21\e{-6} f^2 - 2.03\e{-4} f + 0.928$ \\ \hline
\end{tabular}
\caption{The probabilities of the boundary events as functions of the frequency in GHz.}
\end{table}

From Figure 6, we can see that these probabilities will only make sense until around $520\si{G\hertz}$, since then the rapid rise of the surface decay
rate causes the lambertian probability to decrease sharply. Thus, using the above conversion  between frequency and temperature \eqref{eq:10}, we see that these
distributions are valid for temperatures up to approximately $6\si{K}$.

A further issue for the purposes of the Monte Carlo is that due to an imperfect fit, the probabilities at a given frequency might not sum exactly
to one (being off by around $\sim 0.01$), but this is easily overcome by normalising them by the total which is what we do in this Monte Carlo.

\subsection*{Surface Anharmonic Decay Implementation}
The exact physical mechanisms underlying the surface anharmonic decays aren't well studied, so for the purposes of this Monte Carlo we approximate
the surface behaviour with the bulk behaviour. The distribution of the daughter phonons is determined by the same distributions as the bulk 
energy PDFs and the output direction is randomised.

\section*{Phonon Removal}
Once phonons have downconverted significantly enough, they are removed from the simulation. The threshold of removal is governed by the Aluminium 
energy gap since this relates to whether phonons can be seen by the detectors on the surface. The energy of the phonons is calculated as 
$E = \hbar \omega$ where $\omega = 2\pi \nu$ and $\nu$ is given by \eqref{eq:10}. In particular, the phonon is dropped from the simulation if 
$E < 2E_{gap}$ where $E_{gap} = 1.75\e{-4} \si{\electronvolt}$

In the present implementation, the Monte Carlo is supplied with a `coverage' value, which indicates the proportion of the surface area covered
by aluminium. Whenever the phonon hits a boundary in the simulation, 

\pagebreak
\begin{thebibliography}{99} 
\bibitem[1]{1}
Leman, S.W. Rev.Sci.Instrum. 83 (2012) 091101 

\bibitem[2]{2}
Optimizing the design and analysis of cryogenic semiconductor dark matter detectors for maximum sensitivity - Pyle, Matt Christopher FERMILAB-THESIS-2012-53

\bibitem[3]{3}
Tamura, S. Phys. Rev. B 31, 4 (1985)

\bibitem[4]{4}
Klitsner, T. and Pohl, R. O. Phys. Rev. B 36, 12 (1987)

\bibitem[5]{5}
Trumpp, H. J. and Eisenmenger, W. Z. Physik B 28, 159-171 (1977)
\end{thebibliography}
\end{document}