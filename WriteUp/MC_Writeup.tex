\documentclass[11pt]{article}
\usepackage{amsmath}
\usepackage{amsthm}
\usepackage{amssymb}
\usepackage{enumitem}
\usepackage[margin=0.5in]{geometry}
\usepackage{siunitx}
\usepackage{graphicx}
\graphicspath{ {./} }
\newcommand{\pic}[0]{\textbf{ATTACH PICTURES}}
\newcommand{\DO}{\textbf{DO!!}}
\newcommand{\sig}{On signature sheet.}
\newcommand{\e}[1]{\cdot 10^{#1}}
\newcommand{\vol}{\si{\volt}}
\newcommand{\mv}{\si{m\volt}}
\newcommand{\amp}{\si{\ampere}}
\newcommand{\hz}{\si{\hertz}}
\usepackage[colorlinks]{hyperref}
\usepackage[utf8]{inputenc}

\begin{document}
\title{Phonon Physics and Monte Carlo Techniques in the SuperCDMS Experiment.}
\author{Jason Koeller}
\date{}
\maketitle

\section*{Basic Elements}
Phonons are the quantised vibration modes of crystal lattices. They are described by the Hamiltonian \cite{1}

\begin{equation}
H = \sum_i \frac{p_i^2}{2m} + \sum_{i,j} \frac{m\omega^2}{2}(x_i - x_j)^2
\end{equation}

The aim of this work is to add to the CDMS Monte Carlo by integrating new diffusive propagation and anharmonic
down conversion effects that occur at the boundaries of the crystal lattices in which the phonons propagate. 

For the purposes of this Monte Carlo, we follow the earlier precedent from \cite{1} and treat the phonons as non
interacting particles with specific decay and scattering properties in both the bulk and at the surfaces of the crystal. We
will also be primarily concerned with a low temperature range where acoustic phonons can be modeled by the linear 
dispersion relaion $\omega = vk$. The initial frequencies of the phonons are uniformly distributed from $80 \si{G\hertz}$ to $6.2 \si{T\hertz}$.
This matches the limits set based on the sensitivity of the SuperCDMS experiment. Phonons lower than $80\si{G\hertz}$ will
be not be detected by the TES sensing elements while the high scale is set by the upper bound of the Brillouin zone of
all three modes of vibrations \cite{2}. This means the upper bound, and the initial distribution, for the three modes is  as follows
\begin{table}[!h]
\begin{center}
\begin{tabular}{|c|c|c|} \hline
Mode & Upper Frequency [$\si{T\hertz}$] & Initial Proportion \\  \hline
Longitudinal & $6.2$ & $10\%$ \\ \hline
Fast Transverse & $3.7$ & $35\%$ \\ \hline
Slow Transverse & $2.4$ & $55\%$\\ \hline
\end{tabular}
\end{center}
\caption{The upper frequencies for the initial phonon distribution, and the initial phonon distribution. The initial distribution of phonon modes is as given in both \cite{1} and \cite{2}.}
\end{table}

Computationally, the Monte Carlo model follows a Discrete Event Simulation (DES). We opt for this because simulating discrete time steps of
a given fixed small length is computationally inefficient as a large amount of simulated steps will not be of interest. Instead, we simulate only 
events of interest. To determine which event occurs next, we first calculate a characteristic time step for each process given by 
$\tau = -\frac{ln(r)}{R}$ where $r$ is randomly distributed from $0$ to $1$ and $R$ is the rate of the process in question. In addition, we 
calculate a time associated with hitting the boundary. This time is calculated assuming the phonon travels at its current velocity and thus a simple
calculation is used to find the distance to the closest point on the boundary in the direction of the velocity vector, and then $\tau = \frac{l}{v}$
where $l$ is the calculated distance and $v$ is the current phonon speed. Whichever time is the smallest from these is the process that 
is chosen to occur next, and the global time of the simulation is incremented by that much. The rest of this note will address exactly what processes
we simulate in bulk and on the boundary and how these rates are calculated.

\section*{Bulk Interactions}

The bulk interactions are dominated by two main processes. These are isotopic scattering and bulk anharmonic decay. 

\subsection*{Isotopic Scattering}
As explained in \cite{1}, phonons scatter off mass defects in the crystal. The rate for this process is given by
\begin{equation}
\Gamma_I = B\nu^4
\end{equation}

Where $B = 2.43 \times 10^{-42}$ in Silicon. For the purposes of the simulation, this process involves the incoming phonon spontaneously 
changing its propagation direction in a uniform manner. The new uniform velocity is picked in spherical coordinates, with the azimuthal angle a
uniform choice between $0$ and $2\pi$ and the polar angle chosen from a $\sin(\theta)$ distribution for $\theta \in [0, \pi]$.

\subsection*{Anharmonic Decay}
A second, weaker process in the bulk decay spectrum is anharmonic decay, through which a phonon will down convert to two lower energy
phonons. This occurs only for longitudinal phonons, and the possible decay modes are either $L\rightarrow T+T$ or $L\rightarrow T+L$.

\subsubsection*{$L\rightarrow L'+T$:}
The decay rate for this process is given by \cite{3}

\begin{equation}
\Gamma_{L\rightarrow L'T} \sim \frac{1}{x^2}(1-x^2)^2\left[(1+x)^2 - \delta^2 (1-x)^2\right]\left[1 + x^2 - \delta^2(1-x)^2\right]^2
\end{equation}

Where $x = \frac{\omega_{L'}}{\omega_L}$ is the ratio of the final longitudinal frequency to the initial, $\delta = \frac{v_l}{v_t}$ is the ratio of
the longitudinal phonon velocity in the crystal to transverse phonon velocity and we must have that $\frac{\delta - 1}{\delta + 1} < x < 1$. This 
distribution is used to sample the frequency of the produced longitudinal phonon, and then the transverse phonon frequency is simply set to be
the difference of the initial frequency and the final longitudinal frequency to respect conservation of energy. 

4-momentum conservation also gives us the angles relative to the initial velocity vector of the final phonons:

\begin{equation}
\cos(\theta_{L'}) = \frac{1 + x^2 - \delta^2(1-x)^2}{2x} 
\end{equation}
\begin{equation}
\cos(\theta_T) = \frac{1 - x^2 + \delta^2(1-x)^2}{2\delta(1-x)}
\end{equation}

We plot the (unnormalised) probability distributions for this process below. 

\begin{figure}[!h]
\begin{center}
\includegraphics[scale=.5]{./bulk_anharmonic_pdfs.png}
\end{center}
\caption{The probability densities (unnormalised) of the L (blue) and T(green) phonons of the LLT mode and of the T phonon in the LTT mode (red).}
\end{figure}

\subsubsection*{$L\rightarrow T + T$:}
The decay rate for this process is also given in \cite{3}:

\begin{equation}
\Gamma_{L\rightarrow TT} \sim (A + B\delta x - Bx^2)^2 + \left[Cx(\delta - x) - \frac{D}{\delta - x}\left(x - \delta - \frac{1-\delta^2}{4x}\right)\right]
\end{equation}

Where the constant $A, B, C, D$ are related to the properties of the crystal and can be found in both \cite{2} and \cite{3}. Here, the limits
for the pdf support are $\frac{\delta - 1}{2} \leq x \leq \frac{\delta + 1}{2}$, and the variable itself is now defined as $x = \delta\frac{\omega}
{\omega_0}$. The corresponding angular equations can be obtained once again from
4-momentum conservation 

\begin{equation}
\cos(\theta_1) = \frac{1 - \delta^2 + 2x\delta}{2x}
\end{equation}
\begin{equation}
\cos(\theta_2) = \frac{1 - x\cos(\theta_1)}{d - x}
\end{equation}

These equations are slightly different to those from \cite{1}. The equations presented there are problematic since they are not strictly bounded
by $\pm 1$. The $\cos(\theta)$ distributions calculated here are shown below, adjusted for the scaling of the $x$ variable so that the distributions
are now functions of just the frequency ratio without the scaling factor.

\begin{figure}[!h]
\centering
\includegraphics[scale=.5]{./LTT_cos_theta.png}
\caption{The distribution of the cosine of the scattering angle for the LTT case.}
\end{figure}

As expected, the distributions are very symmetric since both outputs are transverse phonons and thus neither should have a privileged position with 
respect to the other. We have also put the plots from \cite{2} for comparison. The bounds are determined by those for the LTT support and are
the same as those for the red LTT plot in figure 1 above. Note how the previous equations are not bounded properly in this range and 
demonstrates an asymptote at the point where $\frac{\omega}{\omega_0} = \frac{1}{d} = 61$ for silicon.

\pagebreak
Finally, we present (normalised) distributions of the scattering angles of phonons from all interactions:

\begin{figure}[!h]
\centering
\includegraphics[scale=.6]{scattering_angle_distributions.png}
\end{figure}

The LLT distributions match those given in \cite{1}, which serves as a check on our work. The LTT distributions are virtually identical, as expected,
and almost appear to match those shown before, which is strange given the differences in the cosine formulas quoted there from ours. 

\section*{Boundary Interactions}
The key new addition to the Monte Carlo are the interactions at the boundary. Recall that we use DES to simulate our Monte Carlo. If a step terminates
on the boundary then the next step will not choose from the rates for bulk processes but for the following boundary processes:

\begin{enumerate}
\item Specular scattering
\item Lambertian scattering
\item Surface anharmonic decay (SAD)
\end{enumerate}



\pagebreak
\begin{thebibliography}{99} 
\bibitem[1]{1}
Leman, S.W. Rev.Sci.Instrum. 83 (2012) 091101 

\bibitem[2]{2}
Optimizing the design and analysis of cryogenic semiconductor dark matter detectors for maximum sensitivity - Pyle, Matt Christopher FERMILAB-THESIS-2012-53

\bibitem[3]{3}
Tamura, S. Phys. Rev. B 31, 4 (1985)
Optimizing the design and analysis of cryogenic semiconductor dark matter detectors for maximum sensitivity - Pyle, Matt Christopher FERMILAB-THESIS-2012-53
\end{thebibliography}
\end{document}